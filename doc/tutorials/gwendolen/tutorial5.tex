\documentclass[a4]{article}
\usepackage{amssymb}
\usepackage{amsmath}
\usepackage{listings}
\usepackage{graphicx}
\usepackage{../../manual/manual}
\newcounter{lst}[section]

\renewcommand{\thelst}{\arabic{section}.\arabic{lst}}

\lstdefinestyle{easslisting}{basicstyle=\normalfont\footnotesize\sffamily, mathescape=true, numbers=right, numberstyle=\footnotesize, stepnumber=1, numbersep=-5pt, captionpos=b}

\lstdefinestyle{eass}{basicstyle=\sffamily, mathescape=true}

\lstnewenvironment{listing}[3]{
  \noindent        
  \refstepcounter{lst}         
  \label{code:#1}  
\begin{tabular}{p{.97\columnwidth}} \\ \hline  {\normalsize \textbf{Code  fragment \arabic{section}.\arabic{lst}} #2} \\  \end{tabular} 
\lstset{language=#3,          
  basicstyle=\footnotesize\sffamily,
%%%  basicstyle=\footnotesize,    
  xleftmargin=10pt,    
  mathescape=true,    
  frame=tb,    
  numbers=right,    
%%%  numberstyle=\tiny,     
  numberstyle=\footnotesize,     
  stepnumber=1,     
  numbersep=-5pt}}{}

%-- definition for Gwendolen --%

\lstdefinelanguage{Gwendolen}{%
    morekeywords={Plans,Initial,Beliefs,Goals,name,fof-parse,Rules,Belief,Reasoning},
    morecomment=[l]{//},
 literate= {<-}{{$\leftarrow$}{$\:$}}2
           {.B}{{${\cal B}$}}2
           {.G}{{${\cal G}$}}2
           {lnot}{{$\sim$}}2
           {assert_shared}{{$+_{\Sigma}$}}2
           {remove_shared}{{$-_{\Sigma}$}}2
           {(perform)}{}0
}



\makeindex

% table of stacks:
\newenvironment{stacks}{
\smaller
\begin{tabular}{l|l|l|l} 
\hline
event & guard & deed & unifier \\ 
\hline \hline} {
\cr \hline 
\end{tabular}
}

\newenvironment{vplan*}[1][] {
\smaller
\begin{tabular}
        {|l|l|} 
        \hline } {
        \cr \hline 
\end{tabular}
} 

\lstset{basicstyle=\sffamily}
\author{Louise A. Dennis}

\title{Gwendolen Tutorial 5 -- Events and Intentions}

\begin{document}
\maketitle
This is the fourth in a series of tutorials on the use of the \gwendolen\ programming language.  This tutorial looks in more depth at the \gwendolen\ concepts of Event and Intention.  It is primarily theoretical but these will be important concepts for future tutorials.

For this tutorial we will be working with files from previous tutorials.  You may wish to create a separate folder, \texttt{tutorial5} for this work and copy files into it.  Remember to update the paths in your configuration files if you do so.

\gwendolen\ does not have its own debugger, however you can get a long way using error outputs and logging information.

\section{\ail\ -- The Agent Infrastructure Layer}

\gwendolen\ is implemented using the \ail\ Toolkit.  This is mostly irrelevant to the programming of \gwendolen\ agents, however it can be useful in understanding some of the logging output that you may wish to use for debugging, since the logging is based around the underlying \java\ data structures rather than their specific implementation in \gwendolen.

The following discussion of intentions in the \ail\ is taken from~\cite{springerlink:10.1007/s10515-011-0088-x}.

\section{Intentions}

\ail's most complex data structure is that which represents an
\emph{intention}\index{intention}.  BDI languages use intentions to
store the \emph{intended means} for achieving goals -- this is
generally represented as some from of {\em deed stack} (deeds include
actions, belief updates, and the commitment to
goals)\index{deed!stack}.  Intention structures in BDI languages may
also maintain information about the (sub-)goal they are intended to
achieve or the event that triggered them. In \ail, we aggregate
this information: an intention becomes a stack of tuples of an
event\index{event}, a guard\index{guard}, a deed\index{deed}, and a
unifier\index{unifier}.  This \ail{} intention data structure is most
simply viewed as a matrix structure consisting of four columns in
which we record events (new perceptions, goals committed to and so
forth), deeds (a plan of future actions, belief updates, goal
commitments, etc.), guards (which must be true before a deed can be
performed) and unifiers. These columns form an event stack, a deed
stack, a guard stack, and a unifier stack.  Rows associate a
particular deed with the event that has caused the deed to be placed
on the intention, a guard which must be believed before the deed can
be executed, and a unifier. New events are associated with an empty
deed, $\AILnpy$\index{$\AILnpy$}\index{empty deed}\index{no plan yet}.

\paragraph{Example} The following shows the full structure for a
single intention to clean a room.  We use a standard BDI syntax: $!g$
to indicate the goal $g$, and $+!g$ to indicate the commitment to
achieve that goal (i.e., a new goal that $g$ becomes true is
adopted). Constants are shown starting with lower case letters, and
variables with upper case letters.
\begin{center}
        \begin{stacks}
                \inlinecode{+!clean()} & 
                \inlinecode{dirty(Room)} & 
                \inlinecode{+!goto(Room)} &
                \inlinecode{Room = room1} \\
                \inlinecode{+!clean()} & 
                $\top$ & 
                \inlinecode{+!vacuum(Room)} & 
                \inlinecode{Room = room1} 
        \end{stacks}
\end{center}
This intention has been triggered by a desire to clean --- the
commitment to the goal
$clean()$ is the trigger event for both rows in the intention.  An
intention is processed from top to bottom so we see here that the
agent first intends to commit to the goal $goto(Room)$, where $Room$
is to be unified with $room1$.  It will only commit to this goal if it
believes the (guard) statement, $dirty(Room)$.  Once it has committed
to that goal it then commits to the goal $\mathit{vacuum}(Room)$.  In many
languages the process of committing to a goal causes an expansion of
the intention stack, pushing more deeds on it to be processed.  So
$goto(Room)$ may be expanded
\emph{before} the agent commits to vacuuming the room.  In which case
the above intention might become
\begin{center}
        \begin{stacks}
                \inlinecode{+!goto(Room)} &
                $\top$ & 
\inlinecode{+!planRoute(Room, Route)} &
\inlinecode{Room = room1} \\
                \inlinecode{+!goto(Room)} &
$\top$ & 
\inlinecode{+!follow(Route)} & 
\inlinecode{Room = room1} \\
                \inlinecode{+!goto(Room)} & 
                $\top$ & 
\inlinecode{+!enter(Room)} &
\inlinecode{Room = room1} \\
                \inlinecode{+!clean()} & 
$\top$ & 
\inlinecode{+!vacuum(Room)} & 
\inlinecode{Room = room1} 
        \end{stacks}
\end{center}

At any moment, we assume there is a \emph{current intention} which is
the one being processed at that time.  The function
\AILselectintention\ (implemented as a method in \ail) may be used
to select an intention.  By default, this chooses the first intention
from a queue, but this choice may be overridden for specific languages
and applications.  Intentions\index{intention} can be
\emph{suspended}\index{intention!suspended} which allows further
heuristic control.  A suspended intention is, by default, \emph{not}
selected by \AILselectintention.  Typically an intention will remain
suspended until some trigger condition occurs, such as a message being
received.  Many operational semantic rules (such as those involved
with perception) resume\index{intention!resuming} \emph{all}
intentions --- this allows suspension conditions to be re-checked.

\section{Events}

Events are things that occur within the system to which an agent may
wish to react.  Typically we think of these as changes in
beliefs\index{beliefs!change} or the new commitment to
goals\index{goals!commitment}.  In many (though not all) programming
languages, events trigger plans (i.e., a plan might selected for
execution only when the corresponding event has taken place).

In \ail{} there is a special event, `\AILstart{}', that is used to
start off an intention which is not triggered by anything specific.
This is mainly used for the initial goals of an agent --- the
intention begins as a \texttt{start} intention with the deed to commit
to a goal\index{goals!initial}.  In some languages the belief changes
caused by perception are also treated in this way.  Rather than being
added directly to the belief base, in \ail{} such beliefs are assigned
to intentions with the event \AILstart{} and then added to the belief
base when the intention is actually executed.

\section{Intentions in \gwendolen}

Let us look back at some of the logging output generated as part of tutorial 4.

\begin{verbatim}
ail.semantics.AILAgent[FINE|main|4:10:45]: Applying Handle Add Achieve Test Goal with Event 
ail.semantics.AILAgent[FINE|main|4:10:45]: robot
=============
After Stage StageD :
[square/2-square(1,1), square(1,2), square(1,3), square(1,4), square(1,5), square(2,1), square(2,2), square(2,3), square(2,4), square(2,5), square(3,1), square(3,2), square(3,3), square(3,4), square(3,5), square(4,1), square(4,2), square(4,3), square(4,4), square(4,5), square(5,1), square(5,2), square(5,3), square(5,4), square(5,5), ]
[all_squares_checked/0-[_aall_squares_checked()]]
[]
source(self):: 
   *  +!_aall_squares_checked()||True||npy()||[]
   *  start||True||+!_aall_squares_checked()()||[]
[] 
\end{verbatim}
In the above agent state you can see the belief base (beliefs about squares), goal base (\texttt{all\_squares\_checked}), the empty sent messages box and then the current intention.  The main addition is the record of a \emph{source} for the intention (in this case \texttt{self} -- which means the agent generated the intention itself rather than getting it via perception).

As you can see \gwendolen\ uses the \texttt{start} event mentioned above for new intentions.  In this case it was the intention to commit to the achieve goal \lstinline{!all_squares_checked}.  When the goal was committed to it became an event and was placed as a new row on the top of the intention.  The row associated with the \texttt{start} event remains on the intention because this will force the agent to check if the goal is achieved when it reaches that row again.  A special deed has been used \texttt{npy()} which stands for \emph{no plan yet}.  This means that although the goal has been committed to the agent has not yet looked for an applicable plan for achieving the goal.

If you run \texttt{pickuprubble\_ex5.2.gwen} with logging for \texttt{ail.semantics.AILAgent} set to fine then you will later see the intention become:
\begin{verbatim}
source(self):: 
   *  +!_aall_squares_checked()||True||move_to(X,Y)()||[X-1, X0-1, Y-1, Y0-1]
   *  start||True||+!_aall_squares_checked()()||[]
\end{verbatim}
when an applicable plan is found the intention becomes

\begin{verbatim}
source(self):: 
   *  +!_aall_squares_checked()||True||move_to(X,Y)()||[X-1, X0-1, Y-1, Y0-1]
      +!_aall_squares_checked()||True||do_nothing()||[X-1, X0-1, Y-1, Y0-1]
   *  start||True||+!_aall_squares_checked()()||[]
\end{verbatim}
So now the intention is to first take a \texttt{move\_to} action in order to get to (1, 1) and then make a \texttt{do\_nothing} action and then check if the goal has been achieved.

After the agent performs the move action new information comes in from perception.

\begin{verbatim}
source(percept):: 
   *  start||True||+at(1,1)()||[]

[source(self):: 
   *  +!_aall_squares_checked()||True||do_nothing()||[]
   *  start||True||+!_aall_squares_checked()()||[]
] 
\end{verbatim}
The first intention in this list is the \emph{current intention} which is the one the agent will handle next.  In this case it is a new intention (indicated by the \texttt{start} event) and the intention is to add the belief, \texttt{at(1, 1)}.  The source of this intention is noted as \texttt{percept} (i.e., perception) rather than the agent itself.

Since the agent hadn't finished processing the existing intention this is now contained in a list of other intentions.

When the agent adds the new belief the current intention becomes empty, but \gwendolen\ actually adds yet another new intention indicating that a new belief has been adopted which allows the agent to react to this with a new plan.  So the agent's intentions become
\begin{verbatim}
source(percept):: 

[source(self):: 
   *  +!_aall_squares_checked()||True||do_nothing()||[]
   *  start||True||+!_aall_squares_checked()()||[]
, source(self):: 
   *  +at(1,1)||True||npy()||[]
] 
\end{verbatim}
The current intention is empty, and there are now two intentions waiting for attention.  The empty intention will be removed as the agent continues processing.

\gwendolen\ works on each intention in turn handling the top row on the intention.  So the very first intention becomes the current intention again in due course:
\begin{verbatim}
source(self):: 
   *  +!_aall_squares_checked()||True||do_nothing()||[]
   *  start||True||+!_aall_squares_checked()()||[]

[source(self):: 
   *  +at(1,1)||True||npy()||[]
] 
\end{verbatim}

After the agent has done nothing, therefore, the new intention triggered by the new belief \texttt{at(1, 1)} becomes the current intention:
\begin{verbatim}
source(self):: 
   *  +at(1,1)||True||npy()||[]

[source(self):: 
   *  start||True||+!_aall_squares_checked()()||[]
] 
\end{verbatim}

If there was no plan for reacting to the new belief the agent would just delete the intention but since there is a plan the intention becomes
\begin{verbatim}
source(self):: 
   *  +at(X0,Y0)||True||+checked(X0,Y0)()||[X-1, X0-1, Y-1, Y0-1]

[source(self):: 
   *  start||True||+!_aall_squares_checked()()||[]
] 
\end{verbatim}
And so on.

\subsection{Exercises}
Run some of your existing programs with logging of \texttt{ail.semantics.AILAgent} set to fine and see if you can follow how the agent is handling events, intentions and plans.

\bibliography{tutorials}
\bibliographystyle{plain}

\end{document}
