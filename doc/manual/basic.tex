\chapter{Running and Model Checking an Agent Program}
\label{chap:running}

All of the languages implemented using the Agent Infrastructure Layer (AIL)\index{AIL} come provided with a parser which allows files written in that language to be read in and executed.  Examples of programs can be found in the \texttt{src/examples} directory and the tutorials for some of the languages can be found in Part 2 of this manual.

However these languages only describe the agents and these agents must execute within an multi-agent system\index{multi-agent system} consisting of an environment\index{environment} and one or more agents.  Therefore, any specific example needs to first construct such a multi-agent system.  The languages implemented in the AIL all come with classes for parsing input files to sets of agents and many use the \texttt{DefaultEnvironment}\index{environment!default}\index{DefaultEnvironment (class)} class that come with the AIL. Configuration files\index{configuration!AIL}\index{AIL!configuration} can be used to describe the classes necessary  for a given multi-agent system and the class \texttt{ail.mas.AIL}\index{AIL!class} will build and run a multi-agent system from a configuration file.  This can also be invoked using the \texttt{run-AIL}\index{run-AIL} Eclipse Run Configuration.\index{Eclipse}

Model Checking an agent system uses standard JPF\index{JPF!configuration}\index{configuration!JPF} configuration files which can be supplied to the \texttt{gov.nasa.tool.RunJPF}\index{RunJPF (class)} command at the command line, or to the \texttt{run-JPF (MCAPL)}\index{run-JPF} Run Configuration in Eclipse.  These configuration files should incorporate the line \texttt{@using = mcapl} at the top which should ensure that the correct listeners, etc., for AJPF are configured.  This assumes that the MCAPL project has been added to the JPF \texttt{site.properties} as described in the Installation Instructions (Chapter~\ref{chap:installation}).  A general support class for model checking agent systems configured using an AIL configuration file has been provided.  This is \texttt{ail.util.AJPF\_w\_AIL}\index{AJPF\_w\_AIL (class)}. 

\section{Example: Executing a Multi-Agent-System (UNIX Based Systems)}
{\bf Important:} Make sure that \texttt{\ajpfversion/bin} is in your CLASSPATH.  You also need to make sure that the following jar files appear in you CLASSPATH\index{Java (class)path}\index{Unix}

\begin{verbatim}
lib/3rdparty/ev3classes.jar
lib/3rdparty/ev3tools.jar
lib/3rdparty/junit-4.10.jar
lib/3rdparty/eis-0.5.0.jar
lib/3rdparty/java-prolog-parser.jar
lib/3rdparty/jpf.jar
lib/3rdparty/system-rules-1.16.0.jar
\end{verbatim}

We also recommend setting the environment variable AJPF\_HOME\index{AJPF\_HOME} to be the path to \texttt{\ajpfversion} (NB. throughout this chapter, if you are using the development version \texttt{\ajpfversion} should be \texttt{mcapl}).

You can run the simple agent \texttt{pickupagent.gwen} whose code you will find in \texttt{/src/examples/gwendolen/simple/PickUpAgent}\index{example!pickupagent} by calling

\noindent \begin{lstlisting}[basicstyle=\tiny]
> java ail.mas.AIL {$path_to}/src/examples/gwendolen/simple/PickUpAgent/PickUpAgent.ail
\end{lstlisting}
\medskip

\noindent where \texttt{\${path-to}} is the path to ajpf in your system.

You should see output similar to:

\noindent \begin{lstlisting}
 Jun 25, 2012 11:57:24 AM ajpf.util.AJPFLogger info
 INFO: loading property file: /Users/lad/Eclipse/ajpf2018/src/examples/gwendolen/simple/PickUpAgent/PickUpAgent.ail
 done
\end{lstlisting}
\medskip
 
\texttt{PickUpAgent.ail} is a Configuration File\index{AIL!configuration}\index{configuration!AIL} which describes how AIL should build the relevant multi-agent system.

{\bf NB}  If you do not have AJPF\_HOME set then you will need to run the example from the \texttt{\ajpfversion} directory.

\subsection{In Eclipse}

If you have successfully imported the project into Eclipse then you should have two run configurations \texttt{run-AIL}\index{run-AIL} and \texttt{run-JPF (MCAPL)}\index{run-JPF}.  If you do not these can be found in the \texttt{eclipse} sub-directory.\index{Eclipse}

Select the configuration file you will to run, and then select \texttt{run-AIL} from Eclipse's Run Configuration menu.


\section{Example: Model Checking a Multi-Agent System (UNIX Systems)}
\index{Unix}

To verify a multi-agent system, you will need to run \jpf which uses a class contained in \texttt{lib/3rdparty/jpf.jar}.  Make sure this is on your class path.  Call \emph{in the \ajpfversion} directory:

\noindent \begin{lstlisting}[basicstyle=\tiny]
> java gov.nasa.jpf.tool.RunJPF ${path-to}/src/examples/gwendolen/simple/PickUpAgent/PickUpAgent.jpf
\end{lstlisting}\index{RunJPF (class)}\index{example!pickupagent}
\medskip

\noindent where \texttt{\${path-to}} is the path to \ajpfversion\ in your system.

You should see output similar to: 

\noindent \begin{lstlisting}
----------------------------------- search started
      [skipping static init instructions]
JavaPathfinder core system v8.0 (rev 32) - (C) 2005-2014 United States Government. All rights reserved.


====================================================== system under test
ail.util.AJPF_w_AIL.main("/src/examples/gwendolen/simple/PickUpAgent/PickUpAgent.ail","/src/examples/gwendolen/simple/PickUpAgent/PickUpAgent.psl","0")

====================================================== search started: 28/04/17 14:42
		 # choice: gov.nasa.jpf.vm.choice.ThreadChoiceFromSet {id:"ROOT" ,1/1,isCascaded:false}
		 # garbage collection
[INFO] Adding 0 to []
----------------------------------- [1] forward: 0 new
		 # choice: gov.nasa.jpf.vm.choice.IntChoiceFromSet[id="NewAgentProgramState",isCascaded:false,>0]
		 # garbage collection
[INFO] Adding 1 to [0]
----------------------------------- [2] forward: 1 new
		 # choice: gov.nasa.jpf.vm.choice.IntChoiceFromSet[id="NewAgentProgramState",isCascaded:false,>0]
		 # garbage collection
[INFO] Adding 2 to [0, 1]
----------------------------------- [3] forward: 2 new
		 # choice: gov.nasa.jpf.vm.choice.IntChoiceFromSet[id="NewAgentProgramState",isCascaded:false,>0]
		 # garbage collection
[INFO] Adding 3 to [0, 1, 2, 3]
[INFO] Always True from Now On
----------------------------------- [4] forward: 3 visited
----------------------------------- [3] backtrack: 2
----------------------------------- [3] done: 2
----------------------------------- [2] backtrack: 1
----------------------------------- [2] done: 1
----------------------------------- [1] backtrack: 0
----------------------------------- [1] done: 0
----------------------------------- [0] backtrack: -1
----------------------------------- [0] done: -1
----------------------------------- search finished

====================================================== results
no errors detected

====================================================== statistics
elapsed time:       00:00:00
states:             new=3,visited=1,backtracked=4,end=0
search:             maxDepth=3,constraints=0
choice generators:  thread=1 (signal=0,lock=1,sharedRef=0,threadApi=0,reschedule=0), data=3
heap:               new=7563,released=4373,maxLive=3134,gcCycles=4
instructions:       905783
max memory:         155MB
loaded code:        classes=319,methods=4995

====================================================== search finished: 28/04/17 14:42
\end{lstlisting}

\subsection{In Eclipse}

In eclipse you should be able to select \texttt{run-JPF (MCAPL)}\index{run-JPF} from the Run menu while you have \texttt{src/examples/gwendolen/simple/PickUpAgent/PickUpAgent.jpf} selected.  This should generate similar output to the above.

\section{Executing and Model Checking Multi-Agent Systems on Windows Systems}

AJPF and the AIL have not been extensively tested on Windows systems.  In particular all the examples assume UNIX conventions for path names.  \emph{In theory} however, it should be possible to adapt these to Windows systems simply by converting paths to use Windows\index{Windows} style paths.
