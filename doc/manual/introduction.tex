\chapter{Introduction}

\begin{sloppypar}
The \mcapl{} framework is a suite of tools for building interpreters
for agent programming languages and  model checking programs
executing in those interpreters.  It consists of the \ail\ toolkit\index{AIL} for building interpreters for rational agent programming languages (BDI languages) as introduced by~\cite{rao:92a} and the \ajpf\ model checker~\cite{MCAPL_journal}.  \ajpf\ extends the JavaPathfinder\index{JPF} (\jpf) model checker~\cite{VisserHBPL03}\index{AJPF} to prove LTL properties of BDI agents.  This distribution also contains a number of programming languages implemented in the \ail.  Chief among these are \gwendolen~\cite{dennis17gwen}\index{Gwendolen}, the EASS\index{EASS} variant of \gwendolen\ that can be used to program hybrid autonomous systems and GOAL~\cite{GOAL01}\index{GOAL}.  These languages are described in this manual.  It also contains the systems outlined in~\cite{Dennis2015} (\texttt{ethical\_gwen}), \cite{dennis15:_towar_verif_ethic_robot_behav} (\texttt{actiononly} and \texttt{ethical\_governor})\index{ethical reasoning}, \cite{ferrando18} (\texttt{monitor})\index{runtime verification}, \cite{bremner19} (\texttt{pbdi})\index{BDIPython} and a reimplemention of the {\sc HERA} system~\cite{hera} (\texttt{hera}) together with a BDI-style wrapper for it (\texttt{juno}).  These systems are not documented in this manual.
\end{sloppypar}

This manual consists primarily of basic installation instructions and then a set of tutorials covering various aspects of using the system and some of the languages shipped with it.  These tutorials can also be found in the tutorials sub-directory.  Chapter~\ref{chap:installation} provides installation instructions and Chapter~\ref{chap:running} provides simple instructions for running and model-checking a program in the framework.  Chapter~\ref{chap:ajpf} provides a tutorial based description of the use of \ajpf\ for model checking.  Chapter~\ref{chap:ail} describes the use of the \ail\ with particular reference on its use in creating environments for multi-agent systems\index{environment}\index{multi-agent system}\index{AIL}.  Chapter~\ref{chap:gwendolen} is a tutorial introduction to the \gwendolen\ programming language\index{Gwendolen} and chapter~\ref{chap:eass} is a tutorial introduction to its EASS variant\index{EASS} that can be used for programming hybrid autonomous systems~\index{autonomous systems}.  Chapter~\ref{chap:gwendolen_semantics} provides an operational semantics for \gwendolen\ which may, among other things, be useful for people considering implementating their own language in the \ail\ toolkit.  Chapter~\ref{chap:goal} consists of a discussion of the differences between using the \ail\ implementation of the GOAL programming language\index{GOAL} and the version described in~\cite{goalmanual} and a discussion on the use of \ajpf\ to model check \goal\ programs.  \cite{goalmanual} is included in the distribution so that full instructions on programming agents in GOAL are available.

\paragraph{How to read this manual}  If you are installing the \mcapl\ framework then we recommend you start by reading chapters~\ref{chap:installation} and~\ref{chap:running} and follow the instructions to install the system and run some basic programs to check it is functioning correctly.  The other chapters are mostly stand alone and will refer you to the relevant parts of this manual where they are not.  The exception is chapter~\ref{chap:eass} which assumes familiarity with programming in \gwendolen\ (chapter~\ref{chap:gwendolen}).  If your interest is in learning BDI programming in \gwendolen\ or its EASS variant then we recommend you start with chapter~\ref{chap:gwendolen} and proceed to chapter~\ref{chap:eass} if desired.  Chapter~\ref{chap:ail} is also useful for learning to use this system since it covers the creation of environments\index{environment} for multi-agent systems\index{multi-agent system}.  If your interest is primarily in using the \ajpf\ model checker then we recommend you start by read chapter~\ref{chap:ajpf}.  If you want to model check GOAL programs then you should read chapter~\ref{chap:goal}.

{\bf Note} This release comes packaged with Jar files for NASA's Java Pathfinder (JPF)\index{JPF} tool and a number of other libraries.  The relevant jars, zip files of source code and the NASA license can be found in \texttt{lib/3rdparty}.  The release is configured to run using the version of JPF packaged with it, but this means that some adaptation, particularly of configuration files, may be required to run it on systems where JPF is already installed.

The development of the MCAPL Framework would have been impossible without the financial support of the EPSRC via several grants: Model-Checking Agent Programming Languages (EP/D052548), Engineering Autonomous Space Software (EP/F037201/1), Reconfigurable Autonomy (EP/J011770), Verifiable Autonomy (EP/L024845/1), Robotics and AI for Nuclear (EP/R026084/1) and Future AI and Robotics for Space (EP/R026092/1).  Thanks are also owed to Jomi H\"{u}bner, Koen Hindriks, and Angelo Ferrando.
